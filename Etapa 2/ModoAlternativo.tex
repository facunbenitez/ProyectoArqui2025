\documentclass [15pt,a4paper,twoside]{article}
\usepackage[spanish,shorthands=off]{babel}        % shorhands=off is required for babel french in combination with tikz karnaugh....
\usepackage[utf8x]{inputenc}
\usepackage[T1]{fontenc}
\usepackage{amsmath}
\usepackage{geometry}
\geometry{verbose,a4paper, tmargin=3.5cm,bmargin=3.5cm,lmargin=2.5cm,rmargin=2.5cm,headsep=1cm,footskip=1.5cm}
\usepackage{fancyhdr}
\usepackage{colortbl}
\usepackage[dvipsnames]{xcolor}
\usepackage{tikz -timing}
\usepackage{tikz}
\usetikzlibrary{karnaugh}
\pagestyle{fancy}

\definecolor{LogisimKMapColor0}{RGB}{128,0,0}
\definecolor{LogisimKMapColor1}{RGB}{230,25,75}
\definecolor{LogisimKMapColor2}{RGB}{250,190,190}
\definecolor{LogisimKMapColor3}{RGB}{170,110,40}
\definecolor{LogisimKMapColor4}{RGB}{245,130,48}
\definecolor{LogisimKMapColor5}{RGB}{255,215,180}
\definecolor{LogisimKMapColor6}{RGB}{128,128,0}
\definecolor{LogisimKMapColor7}{RGB}{255,255,25}
\definecolor{LogisimKMapColor8}{RGB}{210,245,60}
\definecolor{LogisimKMapColor9}{RGB}{0,0,128}
\definecolor{LogisimKMapColor10}{RGB}{145,30,180}
\definecolor{LogisimKMapColor11}{RGB}{60,180,175}
\definecolor{LogisimKMapColor12}{RGB}{0,130,203}
\definecolor{LogisimKMapColor13}{RGB}{230,190,255}
\definecolor{LogisimKMapColor14}{RGB}{170,255,195}
\definecolor{LogisimKMapColor15}{RGB}{240,50,230}

\fancyhead{}
\fancyhead[C] {Documento generado por Logisim-evolution en Tue May 13 10:49:43 ART 2025}
\fancyfoot[C] {\thepage}
\renewcommand{\headrulewidth}{0.4pt}
\renewcommand{\footrulewidth}{0.4pt}

\makeatother

\begin{document}
\section{Introducción}
Este documento fue generado por logisim-evolution. Cualquier parte de las fuentes TeX puede ser utilizada en sus propios documentos sin ningún problema. En caso de que desee utilizar todas/partes de estas fuentes TeX generadas, por favor (1) no olvide incluir los paquetes requeridos, y (2) incluya una observación de que esta fuente fue generada por logisim-evolution.
%===============================================================================
\section{Tabla de verdad}
La tabla puede ser demasiado grande para ser mostrada en la página. En el momento de la generación no se realizó ningún cálculo sobre el tamaño de la tabla con respecto a la anchura/altura de la página.
%-------------------------------------------------------------------------------
\subsection{Tabla de verdad compactada}
\begin{center}
\begin{tabular}{cccc|c}
$Z1$&$Z2$&$Z3$&$Z4$&$f1$\\
\hline
$-$&$0$&$0$&$0$&$1$\\
$0$&$0$&$0$&$1$&$0$\\
$0$&$-$&$1$&$-$&$0$\\
$-$&$1$&$0$&$0$&$0$\\
$0$&$1$&$0$&$1$&$1$\\
$1$&$0$&$-$&$1$&$1$\\
$1$&$-$&$1$&$0$&$0$\\
$1$&$1$&$0$&$1$&$0$\\
$1$&$1$&$1$&$1$&$1$\\

\end{tabular}
\end{center}
%-------------------------------------------------------------------------------
\subsection{Tabla de la verdad completa}
\begin{center}
\begin{tabular}{cccc|c}
$Z1$&$Z2$&$Z3$&$Z4$&$f1$\\
\hline
$0$&$0$&$0$&$0$&$1$\\
$0$&$0$&$0$&$1$&$0$\\
$0$&$0$&$1$&$0$&$0$\\
$0$&$0$&$1$&$1$&$0$\\
$0$&$1$&$0$&$0$&$0$\\
$0$&$1$&$0$&$1$&$1$\\
$0$&$1$&$1$&$0$&$0$\\
$0$&$1$&$1$&$1$&$0$\\
$1$&$0$&$0$&$0$&$1$\\
$1$&$0$&$0$&$1$&$1$\\
$1$&$0$&$1$&$0$&$0$\\
$1$&$0$&$1$&$1$&$1$\\
$1$&$1$&$0$&$0$&$0$\\
$1$&$1$&$0$&$1$&$0$\\
$1$&$1$&$1$&$0$&$0$\\
$1$&$1$&$1$&$1$&$1$\\

\end{tabular}
\end{center}
%===============================================================================
\section{Diagramas de Karnaugh}
Esta sección muestra varias versiones de los diagramas de Karnaugh de las funciones dadas.
%-------------------------------------------------------------------------------
\subsection{Diagramas de Karnaugh vacíos}
\begin{center}
\begin{tikzpicture}[karnaugh,disable bars,x=1\kmunitlength,y=1\kmunitlength,kmbar left sep=1\kmunitlength,grp/.style n args={4}{#1,fill=#1!30,minimum width= #2\kmunitlength,minimum height=#3\kmunitlength,rounded corners=0.2\kmunitlength,fill opacity=0.6,rectangle,draw}]
\karnaughmap{4}{$f1$}{{$Z1$}{$Z3$}{$Z2$}{$Z4$}}{}{
\draw[kmbox] (-0.5,4.5)
   node[below left]{$Z1$, $Z2$}
   node[above right]{$Z3$, $Z4$} +(-0.2,0.2)
   node[above left]{$f1$};\draw (0,4) -- (-0.7,4.7);
\foreach \x/\1 in %
{0/00,1/01,2/11,3/10} {
   \node at (\x+0.5,4.2) {\1};
}
\foreach \y/\1 in %
{0/00,1/01,2/11,3/10} {
   \node at (-0.4,-0.5-\y+4) {\1};
}
}
\end{tikzpicture}
\end{center}
%-------------------------------------------------------------------------------
\subsection{Rellenado de los diagramas de Karnaugh}
\begin{center}
\begin{tikzpicture}[karnaugh,disable bars,x=1\kmunitlength,y=1\kmunitlength,kmbar left sep=1\kmunitlength,grp/.style n args={4}{#1,fill=#1!30,minimum width= #2\kmunitlength,minimum height=#3\kmunitlength,rounded corners=0.2\kmunitlength,fill opacity=0.6,rectangle,draw}]
\karnaughmap{4}{$f1$}{{$Z1$}{$Z3$}{$Z2$}{$Z4$}}
{1001000011000101}{
\draw[kmbox] (-0.5,4.5)
   node[below left]{$Z1$, $Z2$}
   node[above right]{$Z3$, $Z4$} +(-0.2,0.2)
   node[above left]{$f1$};\draw (0,4) -- (-0.7,4.7);
\foreach \x/\1 in %
{0/00,1/01,2/11,3/10} {
   \node at (\x+0.5,4.2) {\1};
}
\foreach \y/\1 in %
{0/00,1/01,2/11,3/10} {
   \node at (-0.4,-0.5-\y+4) {\1};
}
}
\end{tikzpicture}
\end{center}
%-------------------------------------------------------------------------------
\subsection{Relleno de diagramas de Karnaugh con tapas}
\begin{center}
\begin{tikzpicture}[karnaugh,disable bars,x=1\kmunitlength,y=1\kmunitlength,kmbar left sep=1\kmunitlength,grp/.style n args={4}{#1,fill=#1!30,minimum width= #2\kmunitlength,minimum height=#3\kmunitlength,rounded corners=0.2\kmunitlength,fill opacity=0.6,rectangle,draw}]
\karnaughmap{4}{$f1$}{{$Z1$}{$Z3$}{$Z2$}{$Z4$}}
{1001000011000101}{
\draw[kmbox] (-0.5,4.5)
   node[below left]{$Z1$, $Z2$}
   node[above right]{$Z3$, $Z4$} +(-0.2,0.2)
   node[above left]{$f1$};\draw (0,4) -- (-0.7,4.7);
\foreach \x/\1 in %
{0/00,1/01,2/11,3/10} {
   \node at (\x+0.5,4.2) {\1};
}
\foreach \y/\1 in %
{0/00,1/01,2/11,3/10} {
   \node at (-0.4,-0.5-\y+4) {\1};
}
   \node[grp={LogisimKMapColor0}{0.8}{0.8}](n0) at(0.5,3.5) {};
   \node[grp={LogisimKMapColor0}{0.8}{0.8}](n1) at(0.5,0.5) {};
   \node[grp={LogisimKMapColor1}{0.8}{0.8}](n2) at(1.5,2.5) {};
   \node[grp={LogisimKMapColor2}{0.8}{1.8}](n3) at(2.5,1) {};
   \node[grp={LogisimKMapColor3}{1.8}{0.8}](n4) at(1,0.5) {};
}
\end{tikzpicture}
\end{center}
%===============================================================================
\section{Expresiones mínimas}
$f1 =  \overline{Z2}  \cdot  \overline{Z3}  \cdot  \overline{Z4} + \overline{Z1}  \cdot Z2 \cdot  \overline{Z3}  \cdot Z4+Z1 \cdot Z3 \cdot Z4+Z1 \cdot  \overline{Z2}  \cdot  \overline{Z3} $~\\
\end{document}
